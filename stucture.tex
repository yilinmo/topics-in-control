\documentclass{article}

\usepackage{subfigure}
\usepackage{amssymb, amsmath, amsfonts}
\usepackage{amsthm}
\usepackage{tikz}
\usetikzlibrary{external}
\usepackage{pgfplots}

\newtheorem{proposition}{Proposition}
\newtheorem{theorem}{Theorem}
\newtheorem{definition}{Definition}
\newtheorem{lemma}{Lemma}
\newtheorem{conjecture}{Conjecture}
\newtheorem{corollary}{Corollary}
\newtheorem{remark}{Remark}
\newtheorem{assumption}{Assumption}

\newlength\figureheight
\newlength\figurewidth

\ifnum\pdfshellescape=1
\tikzexternalize[prefix=tikzpdf/]
\fi

\newcommand{\tikzdir}[1]{tikz/#1.tikz}
\newcommand{\inputtikz}[1]{\tikzsetnextfilename{#1}\input{\tikzdir{#1}}}

\DeclareMathOperator*{\argmin}{arg\; min}     % argmin
\DeclareMathOperator*{\argmax}{arg\; max}     % argmax
\DeclareMathOperator*{\tr}{tr}     % trace
\DeclareMathOperator{\Cov}{Cov}
\DeclareMathOperator{\logdet}{log\;det}

\newif\ifcomment
%display the comments
\commenttrue
%do not display the comments, used for the submitted version
%\commentfalse
\ifcomment
\newcommand{\comment}[1]{\textcolor{blue}{#1}}
\else
\newcommand{\comment}[1]{}
\fi
\title{Generic Properties of Linear Structured System}

\author{Yilin Mo}

\begin{document} \maketitle

Consider the space of $n$ by $n$ matrix $\mathbb R^{n\times n}$. We can define the canonical basis $E_{i,j}$ as an all zero matrix except the $i$th row and $j$th column to be one.

Let us consider a subspace $U\subseteq \mathbb R^{n\times n}$, such that $U = span(E_{i_1,j_1},\dots,E_{i_l,j_l})$.

We know that any matrix $X \in U$ can be written as
\begin{displaymath}
  X = \sum_{k=1}^l\lambda_k E_{i_k,j_k}.
\end{displaymath}

Hence, there is a one to one correspondence between $X$ and $\mathbb R^l$.

We will call $U$ a ``structured matrix''.

Define
\begin{displaymath}
 \det(\Lambda) = \left|\sum_{k=1}^l \lambda_k E_{i_k,j_k}\right| 
\end{displaymath}
\begin{theorem}
  Let $U$ be a ``structured matrix''. Define the set
  \begin{displaymath}
    S = \{\Lambda\in \mathbb R^l:\det(\Lambda) =0\} 
  \end{displaymath}
  Then one of the following statements hold:
  \begin{itemize}
    \item $S = \mathbb R^l$.
    \item $S$ is nowhere dense in $\mathbb R^l$ and has Lebesgue measure $0$. $\mathbb R^l \backslash S$ is open.
  \end{itemize}
\end{theorem}
\begin{proof}
  Suppose that there exists $\Lambda^*$, such that
  \begin{displaymath}
    \det\left( \Lambda^* \right) \neq 0. 
  \end{displaymath}

For any $\Lambda $, consider the following function:
\begin{displaymath}
  f(t) = \det\left[ (1-t)\Lambda + t \Lambda^* \right].
\end{displaymath}
We know that $f(t)$ is at most an $n$th-polynomial of $t$ and $f(0) = \det\left( \Lambda \right)$ and $f(1) = \det(\Lambda^*)\neq 0$.

Therefore, $f\neq 0$ and there is at most $n$ roots for $f(t)$. Hence, for any $r > 0$, there exists a $|t| < r$, such that $f(t)\neq 0$.

Notice that $\det(\Lambda)$ is continuous with respect to $\Lambda$. Hence, $\mathbb R^l\backslash S = \det^{-1}((-\infty,0)\bigcup(0,\infty) )$ is open. Thus, $S$ is closed and has no interior. Thus, $S$ is nowhere dense.

In fact, $S$ is a proper algebraic variety. The $0$ Lebesgue measure of $S$ can be proved using the properties of algebraic variety.
\end{proof}

The above theorem essentially implies that a zero-one phenomenon regarding invertibility:
\begin{itemize}
  \item Either all matrices in $U$ are not invertible;
  \item or almost all matrices in $U$ are invertible.
\end{itemize}

Consider the set of stable matrices in $U$, one can easily prove that in fact some matrices in $U$ are stable, some are not. (assuming that $U$ is non-trivial).

Hence, invertibility only depends on the structure of $U$ (generic properties), while stability depends on the exact parameterization. 

For any ``structured matrix'' $U$, we can define the graph associated with $U$ as $G(U) = (V,E)$, where $V = \{1,\dots,n\}$ and $(i,j)\in E$ if and only if $E_{i,j}\in U$. The graph there is a one-to-one correspondence between $G(U)$ and $U$. Hence, any generic properties can be derived from $G(U)$.

Let us define a simple path as a path without repetitions of vertices or edges. Two paths are disjoint if they do not share any common vertex. A family of paths are disjoint if and only any two paths are disjoint.

Similarly, we can define simple cycle and disjoint cycles. In the remaining lecture, we will only consider simple paths and simple cycles.

\begin{theorem}
  A structured matrix $U$ is generically invertible if and only if there exists a family of disjoint cycles in $G(U)$ that covers all vertices $V$.
\end{theorem}
\begin{proof}
  If $U$ is generically invertible, then there exists an $X\in U$ that is invertible.

  Let $\sigma = (\sigma_1,\dots,\sigma_n)$ be a permutation of $(1,\dots,n)$, we know that
  \begin{displaymath}
    \det(X) = \sum_{\sigma} sgn(\sigma) \prod_{i=1}^n X_{i,\sigma_i}\neq 0
  \end{displaymath}

  Hence, there exists at least one $\sigma$, such that $X_{i,\sigma_i}\neq 0$ for all $i=1,\dots,n$. We claim that the edge set $(i,\sigma_i)$ forms a family of disjoint cycles that covers $V$.

  On the other hand, if the edge set $\{(i_k,j_k)\}$, $k=1,\dots,n$ forms a family of disjoint cycles that covers $V$, then we can define 
\begin{displaymath}
  X_{ij} = \begin{cases}
    1&\text{ if } (i,j)=(i_k,j_k)\text{ for some }k\\
    0&\text{ otherwise}
  \end{cases}
\end{displaymath}
We claim that $\det(X) = 1$ or $-1$. Hence, $U$ must be generically invertible.
\end{proof}


Alternatively, we can construct a bipartite graph $G'(U) = (V',E')$, where $V' = \{i_1,\dots,i_n,o_1,\dots,o_n\}$, such that
\begin{displaymath}
  (i_a,o_b)\in E' \Leftrightarrow E_{a,b}\in U. 
\end{displaymath}

A matching of a graph is defined as a family of disjoint edges. $U$ is generically invertible if and only if there exists a matching with size $n$.

We can generalize the graph $G'(U)$ to non-square matrices. The generic rank of $U$ is given by the size of the maximum matching.
\section{Structured Linear System}
Consider the following linear system
\begin{displaymath}
 x(k+1) = A x(k) + B u(k),\, y(k) = C x(k) + D u(k), 
\end{displaymath}
where $A,B,C,D$ are all structured matrices. $x(k)\in \mathbb R^n$, $y(k)\in \mathbb R^m$ and $u(k)\in \mathbb R^p$. We can define the graph $G(A,B,C,D) = (V,E)$ of the system as
\begin{itemize}
  \item $V = \{u_1,\dots,u_p, x_1,\dots,x_n,y_1,\dots,y_m\}$. 
  \item $(x_i,x_j)\in E$ if and only if $A_{i,j}$ is a free parameter.
  \item $(x_i,u_j)\in E$ if and only if $B_{i,j}$ is a free parameter.
  \item $(y_i,x_j)\in E$ if and only if $C_{i,j}$ is a free parameter.
  \item $(y_i,u_j)\in E$ if and only if $D_{i,j}$ is a free parameter.
\end{itemize}
Any generic properties of the system can be derived from the graph $G(A,B,C,D)$.

Controllability(Observability) is a generic property, since we are checking the invertibility of
\begin{displaymath}
  \mathcal B = \begin{bmatrix}
    B&AB&\dots&A^{n-1}B
  \end{bmatrix}.
\end{displaymath}
Every entries in $\mathcal B$ is a polynomial of parameters. Hence, the determinant of a submatrix is also a polynomial of parameters. By the same argument as the square matrix case, we know that the invertibility of $\mathcal B$ is a generic property.

\begin{theorem}
  If one of the following conditions holds:
  \begin{enumerate}
    \item $rank\begin{bmatrix} A & B\end{bmatrix} < n$, 
    \item There exists a permutation matrix $P$, such that
      \begin{displaymath}
	PAP^{-1} = \begin{bmatrix}
	  A_{11}& 0\\
	  A_{21}&A_{22}
	\end{bmatrix},\, PB = \begin{bmatrix}
	  0\\
	  B_2
	\end{bmatrix}
      \end{displaymath},
  \end{enumerate}
  then $(A,B)$ is not observable.
\end{theorem}
\begin{proof}
  If $\begin{bmatrix} A&B\end{bmatrix}$ is not full rank, then there exists $v\neq 0$, such that
    \begin{displaymath}
     v^TA = v^T B = 0, 
    \end{displaymath}
    which implies that $v^TA^k B = 0$. Hence, $\mathcal B$ is not full rank.

    On the other hand, if the second conditions hold, then we can choose $v = \begin{bmatrix} \mathbf 1 \\0  \end{bmatrix}$, then $v^TA^k = \begin{bmatrix} \mathbf 1^T A_{11} & 0\end{bmatrix}$. Thus, $v^TA^kB = 0$, for all $k$ and $\mathcal B$ is not full rank.
\end{proof}

\begin{theorem}
  The generic rank of $\begin{bmatrix} A&B\end{bmatrix} = n$ if and only if there exists a disjoint family of
    \begin{enumerate}
      \item cycles,
      \item paths that start from some $u_j$,
    \end{enumerate}
    which covers the vertices $x_1,\dots,x_n$.
\end{theorem}

\begin{theorem}
  Condition 2 holds if and only if there exists an vertex $x_i$ that cannot be reached from any $u_j$.
\end{theorem}

Thus we know that
\begin{theorem}
  $(A,B)$ is generically controllable implies that 
  \begin{enumerate}
    \item every vertex $x_i$ can be reached from some $u_j$, 
    \item and all vertices $x_1,\dots,x_n$ are covered by a disjoint family of cycles and paths that starts from some $u_j$.
  \end{enumerate}
  \label{theorem:genericcontrollability}
\end{theorem}
We can prove that the converse in Theorem~\ref{theorem:genericcontrollability} is also true. 

Examples:
\begin{enumerate}
  \item \begin{displaymath}
  A = \begin{bmatrix}
    a_{11}&a_{12}&0\\
    a_{21}&a_{22}&0\\
    a_{31}&a_{32}&a_{33}
  \end{bmatrix},\,B = \begin{bmatrix}
    0\\
    0\\
    b_{31}
  \end{bmatrix}
\end{displaymath}
Not controllable since $x_1,x_2$ are not connected to $u_1$.
\item \begin{displaymath}
  A = \begin{bmatrix}
    0&a_{12}&0\\
    0&a_{22}&0\\
    0&a_{32}&0
  \end{bmatrix},\,B = \begin{bmatrix}
    b_{11}\\
    b_{21}\\
    b_{31}
  \end{bmatrix}
\end{displaymath}
Not controllable since we cannot cover $x_1,x_3$ simultaneously with disjointed paths.
\item \begin{displaymath}
  A = \begin{bmatrix}
    0&a_{12}&0&\dots& 0\\
    0&0&a_{23}&\dots&0\\
    \vdots&\vdots&\vdots&\ddots&\vdots\\
    0&0&0&\dots&a_{n-1,n}\\
    0&0&0&\dots&0\\
  \end{bmatrix},\,B = \begin{bmatrix}
    0\\
    \vdots\\
    b_{n1}
  \end{bmatrix}
\end{displaymath}
Generically controllable. The corresponding graph is called a ``stem''.
\item \begin{displaymath}
  A = \begin{bmatrix}
    0&a_{12}&0&\dots& 0\\
    0&0&a_{23}&\dots&0\\
    \vdots&\vdots&\vdots&\ddots&\vdots\\
    0&0&0&\dots&a_{n-1,n}\\
    a_{n1}&0&0&\dots&0\\
  \end{bmatrix},\,B = \begin{bmatrix}
    0\\
    \vdots\\
    b_{n1}
  \end{bmatrix}
\end{displaymath}
Generically controllable. The corresponding graph is called a ``bud''.
\end{enumerate}


\end{document}




