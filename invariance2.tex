\documentclass{article}

\usepackage{subfigure}
\usepackage{amssymb, amsmath, amsfonts}
\usepackage{amsthm}
\usepackage{tikz}
\usetikzlibrary{external}
\usepackage{pgfplots}

\newtheorem{proposition}{Proposition}
\newtheorem{theorem}{Theorem}
\newtheorem{definition}{Definition}
\newtheorem{lemma}{Lemma}
\newtheorem{conjecture}{Conjecture}
\newtheorem{corollary}{Corollary}
\newtheorem{remark}{Remark}
\newtheorem{assumption}{Assumption}

\newlength\figureheight
\newlength\figurewidth

\ifnum\pdfshellescape=1
\tikzexternalize[prefix=tikzpdf/]
\fi

\newcommand{\tikzdir}[1]{tikz/#1.tikz}
\newcommand{\inputtikz}[1]{\tikzsetnextfilename{#1}\input{\tikzdir{#1}}}

\newcommand{\y}{{\mathcal Y}}
\renewcommand{\u}{{\mathcal U}}
\DeclareMathOperator*{\argmin}{arg\; min}     % argmin
\DeclareMathOperator*{\argmax}{arg\; max}     % argmax
\DeclareMathOperator*{\tr}{tr}     % trace
\DeclareMathOperator{\Cov}{Cov}
\DeclareMathOperator{\logdet}{log\;det}
\DeclareMathOperator{\col}{col}

\newif\ifcomment
%display the comments
\commenttrue
%do not display the comments, used for the submitted version
%\commentfalse
\ifcomment
\newcommand{\comment}[1]{\textcolor{blue}{#1}}
\else
\newcommand{\comment}[1]{}
\fi
\title{Secure Control: Intrusion Detection and Identification}

\author{Yilin Mo}

\begin{document} \maketitle
\section{Fault Detection and Identification}
\subsection{Detection}
Consider the following system:
\begin{align}
  x(k+1) = A x(k) + B u(k),\, y(k) = C x(k).
  \label{eq:ltisystem1}
\end{align}
We assume that $(A,C)$ is observable, $B$ is full column rank. Suppose that $u(k)$ is the fault signal. We will say that a fault occurs when $u(k) \neq 0$ for some $k$.

Define $\y = (y(0),y(1),\dots)$ and $\u = (u(0),u(1),\dots)$. Clearly, $\y$ is a function of $x(0)$ and $\u$. Thus, we will write
\begin{align*}
 y = f(x(0),\u). 
\end{align*}

Fact: $f$ is a linear operator.

\comment{Question: Can we know whether a fault occurs from $y(k)$?}

There are two cases depending whether we know $x(0)$ or not. 

Suppose that $x(0)$ is known. Then the nominal trajectory is given by
\begin{align*}
 \y^* = f(x(0),0). 
\end{align*}

On the other hand, if there exists a $\u\neq 0$, such that 
\begin{align*}
 \y^* = f(x(0),\u), 
\end{align*}
then there is no way for us to know whether there is a fault or not given $y^*$. Notice that
\begin{align*}
 \y^* = f(x(0),0) = f(x(0),\u) \Rightarrow f(0,\u) = 0,
\end{align*}
which gives the following theorem:
\begin{theorem}
  The following statements are equivalent:
  \begin{enumerate}
    \item The fault is detectable with known initial conditions. 
    \item The following implication holds:
      \begin{align*}
	f(0,\u)=0\implies \u=0
      \end{align*}
    \item The system is left-invertible, i.e., the mapping from $\u$ to $\y$ defined by $\y = f(0,\u)$ is one to one.
  \end{enumerate}
  \label{eq:detectabilityknowninit}
\end{theorem}

If the initial condition is unknown, then the nominal trajectory will be a set of
\begin{align*}
  Y^* = \{\y:\y=f(x(0),0)\text{ for some }x(0)\in \mathbb R^n\}. 
\end{align*}
By the similar argument, if there exists a $\u$ and $x(0)'$, such that 
\begin{align*}
 \y = f(x(0)',\u) \in Y^*,
\end{align*}
then there is no way to know whether a fault occurs or not given $\y$. By linearity of $\y$, we know that
\begin{align*}
 \y = f(x(0),0) = f(x(0)',\u) \implies f(x(0)'-x(0),\u) = 0, 
\end{align*}
which leads to the following theorem:
\begin{theorem}
  The following statements are equivalent:
  \begin{enumerate}
    \item The fault is detectable with unknown initial conditions. 
    \item The following implication holds:
      \begin{align*}
	f(x(0),\u)=0\implies x(0)=0\text{ and }\u=0.
      \end{align*}
    \item The system has no non-trivial zero dynamics (strongly observable), i.e.,
      \begin{align*}
	f(x(0),\u) = 0 \implies x(k) = 0,\,\forall k.
      \end{align*}
    \item The system does not have an invariant zero, i.e., there does not exists an $z\in \mathbb C$, and non-zero $x_0\in\mathbb R^n$ and $u_0\in\mathbb R^m$, such that
      \begin{align}
	Ax_0 + B u_0 = zx_0,\text{ and } Cx_0=0.
	\label{eq:invariantzero}
      \end{align}
  \end{enumerate}
\end{theorem}
\begin{proof}
  We will only prove $3\implies 2$. Suppose that there exists $x(0)$ and $\u\neq 0$, such that $f(x(0),\u) = 0$. Let us define the subspace $\mathcal V\in \mathbb R^n$ as
  \begin{align*}
    \mathcal V \triangleq \text{span}(x(0),x(1),\dots,).
  \end{align*}
  $\mathcal V \neq \{0\}$. Since 
  \begin{align*}
   Ax(k) = x(k+1) -  B u(k),
  \end{align*}
 we know that
 \begin{align}
  A\mathcal V \subseteq \mathcal V + \col(B) ,
  \label{eq:controlledinvariant}
 \end{align}
 where $\col(B)$ is the column space of $B$. Furthermore,
 \begin{align*}
  C\mathcal V = 0 \implies \mathcal V\subseteq \ker(C), 
 \end{align*}
 where $\ker(C)$ is the null space of $C$. By \eqref{eq:controlledinvariant}, we know that there exists an $K$, such that
 \begin{align*}
  (A+BK)\mathcal V \subseteq \mathcal V.
 \end{align*}
 Hence, there exists $x_0 \in \mathcal V$, which is an eigenvector of $A+BK$ with corresponding eigenvalue $z$. Define $u_0 = Kx_0$, then $z,x_0,u_0$ satisfies \eqref{eq:invariantzero}.
\end{proof}

\begin{remark}
  The system is called strongly detectable if the following implication holds:
  \begin{align*}
   f(x(0),\u) = 0\implies x(k)\rightarrow 0. 
  \end{align*}
  This implies that even there might exists an undetectable attack, the effect of the attack on the state is decaying over time. One can prove that a system is strongly detectable if and only if all the invariant zeros of the system are stable.
\end{remark}

\subsection{Identification}
Consider the following system:
\begin{align}
  x(k+1) = A x(k) + \sum_{i\in \mathcal I}B_i u_i(k),\, y(k) = C x(k),
  \label{eq:ltisystem2}
\end{align}
where $u_i(k)$ denotes the $i$th fault and we say it occurs if $u_i(k)\neq 0$ for some $k$. We assume that at most one fault occurs and we want to identify which one.

Suppose that $x(0)$ is known, then all possible trajectories generated by the $i$th fault can be written as
\begin{align*}
  Y_i = \{\y:\y = f(x(0), B_i\u_i)\},
\end{align*}
where $B_i \u_i = (B_i u_i(0),B_iu_i(1),\dots)$. We claim that we can distinguish the $i$th fault and the $j$th fault if 
\begin{align*}
 \y = f(x(0),B_i\u_i) =  f(x(0),B_j\u_j)\implies \u_i = \u_j = 0.
\end{align*}
Notice that
\begin{align*}
  f(x(0),B_i\u_i) =  f(x(0),B_j\u_j)\Leftrightarrow f\left(0,\begin{bmatrix}B_i &B_j\end{bmatrix}\begin{bmatrix}\u_i\\-\u_j\end{bmatrix}\right)=0.
\end{align*}
Therefore, the fault is identifiable if and only if for any $i\neq j$, $(A,\begin{bmatrix}B_i&B_j\end{bmatrix},C)$ is left invertible.

  Similarly, with unknown initial conditions, the fault is identifiable if and only if for any $i\neq j$, $(A,\begin{bmatrix}B_i&B_j\end{bmatrix},C)$ has no invariant zeros.

\section{Generic Detectability}

We model a network composed of $m$ agents as a graph $G = \{V,\,E\}$. $V = \{1,2,\ldots,m\}$ is the set of vertices representing the agents. $E \subseteq V\times V$ is the set of edges. $(i,j)\in E$ if and only if $j$ can send information to $i$. \comment{The graph can be directed.}

Define the neighbors $\mathcal N_i$ of agent $i$ as the set of agents who can send information to $i$, i.e.,
\begin{displaymath}
  \mathcal N_i \triangleq \{j:(i,j)\in E,\,j\neq i\}.
\end{displaymath}

Suppose each agent has a state $x_i(t)$. The agent update the state based on the following update equation:
\begin{displaymath}
  x_i(k+1) = a_{ii} x_i(k) + \sum_{j\in \mathcal N_i} a_{ij} x_j(k) + u_i(k),
\end{displaymath}
where $u_i(k)$ is a malicious input. A node is benign if $u_i(k)=0$ for all $k$. It is malicious if $u_i(k)\neq 0$ for some $k$.

We can write the above equation in matrix form as:
\begin{displaymath}
  x(k+1) = A x(k) + Bu(k),
\end{displaymath}
where $B = \begin{bmatrix}e_{i_1},\dots,e_{i_f}\end{bmatrix}$, where $\{i_1,\dots,i_f\}$ are the set of malicious node. Furthermore, for node $i$, we can define
\begin{align*}
  C_i = \begin{bmatrix}
    e_{i_1}\\
    \vdots\\
    e_{i_l} 
  \end{bmatrix},
\end{align*}
where $\mathcal N_i \bigcup \{i\} = \{i_1,\dots,i_l\}$. As a result, for a benign node $i$, it observes
\begin{align*}
 y(k) = C_i x(k).
\end{align*}

One can see that there is a straight forward connection between the topology of the network and the graph associated with linear structured system $(A,B,C_i)$

\begin{theorem}
  (Assuming unknown initial condition:) If the graph $G$ has connectivity $k > f$, and $i$ be a benign node. Then for almost any $A,B$ matrices, node $i$ can detect the existence of a malicious behavior. On the other hand, if $k \leq f$, then there exists a set of malicious node $\{i_1,\dots,i_f\}$, such that no node can detect the malicious behavior for any $A,B$.
\end{theorem}
\end{document}




